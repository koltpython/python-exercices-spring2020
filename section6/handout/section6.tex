\documentclass[a4paper]{article}

\usepackage[english]{babel}
\usepackage[utf8]{inputenc}
\usepackage{amsmath}
\usepackage{graphicx}
\usepackage[colorinlistoftodos]{todonotes}
\usepackage[useregional]{datetime2}
\usepackage{fancyhdr}
\usepackage{titlesec}
\usepackage{listings} 
\usepackage[hidelinks]{hyperref}
\usepackage{caption}
\usepackage{titling}
\usepackage{upquote}
\usepackage{color}

\usepackage{wrapfig}
\usepackage{tikz}
\usetikzlibrary{shapes.symbols,positioning, chains}

\setcounter{secnumdepth}{4}

\setlength{\droptitle}{-4em}

\setlength{\intextsep}{6pt plus 2pt minus 2pt}


%Project Title
\newcommand{\handoutTitle}{Error Handling, File Input & Output}
%Homework Number
\newcommand{\handoutNumber}{1}

% Handout date goes here.
% Format: # Day of Month
\newcommand{\handoutDate}{10th of March}

%Contact to reach.
\newcommand{\contactName}{Ceren Kocaoğullar & Hasan Can Aslan}
\newcommand{\contactMail}{ckocaogullar15, haslan16@ku.edu.tr}

%Define colors as shown below to use in text.
\definecolor{Red}{RGB}{255, 0, 0}
\definecolor{Green}{RGB}{0, 255, 0}
\definecolor{Blue}{RGB}{0, 0, 255}
\definecolor{codegreen}{rgb}{0,0.6,0}
\definecolor{codegray}{rgb}{0.5,0.5,0.5}
\definecolor{codepurple}{rgb}{0.58,0,0.82}
\definecolor{backcolour}{rgb}{0.95,0.95,0.92}

% code listing style
\lstset{language=Python}
\lstdefinestyle{mystyle}{
    backgroundcolor=\color{backcolour},   
    commentstyle=\color{codegreen},
    keywordstyle=\color{magenta},
    numberstyle=\tiny\color{codegray},
    stringstyle=\color{codepurple},
    basicstyle=\ttfamily\footnotesize,
    breakatwhitespace=false,         
    breaklines=true,                 
    captionpos=b,                    
    keepspaces=true,                 
    numbers=left,                    
    numbersep=5pt,                  
    showspaces=false,                
    showstringspaces=false,
    showtabs=false,                  
    tabsize=2
}
\renewcommand{\lstlistingname}{Code snippet}
 
\lstset{style=mystyle}

\pagestyle{fancy}
\fancyhf{}
\chead{\handoutTitle}
\rhead{Handout \#\handoutNumber}
\lhead{KOLT Python}
\lfoot{\nouppercase{\leftmark}}
\rfoot{Page \thepage}
\thispagestyle{fancy}
\renewcommand{\headrulewidth}{0.4pt}
\renewcommand{\footrulewidth}{0.4pt}

\author{\contactName\\\href{mailto:\contactMail}{\contactMail}}

\title{\handoutTitle}

\date{Date: \handoutDate}

\begin{document}

\maketitle

\section{Overview}
% Brief explanation of this section's goal
This handout is prepared for \href{https://koltpython.com}{KOLT Python Certificate Program}. It contains a brief review of this week's topics and exercise questions.\\
You can download the starter code from \href{https://kinolien.github.io/gitzip/?download=koltpython/python-exercises-spring2020/tree/master/section4/starter}{\underline{here}}.\\
You can find the solutions at \href{https://kinolien.github.io/gitzip/?download=koltpython/python-exercises-spring2020/tree/master/section4/solution}{\underline{here}} after all the sections are conducted.

\section{Review}
% Basic review of related material
% Remove this section if you believe it is unnecessary

\subsection{Data Structures}
\subsubsection{Tuples}
\begin{itemize}
    \item \textbf{Immutable} sequence of elements.
    \subitem Cannot make changes on a tuple using functions like add(), append(), remove() etc.
    \item Elements are \textbf{ordered.}
    \subitem Therefore you \textbf{can} use indexing, slicing.
    \subitem You can iterate over them using for loops.
    \item To create an empty one: \texttt{tuple()} or just \texttt{()}.
\end{itemize}

\subsubsection{Sets}
\begin{itemize}
    \item \textbf{Unordered} sequence of elements.
    \item Elements are \textbf{unique}.
    \subitem Therefore you \textbf{cannot} use indexing/slicing.
    \subitem But you \textbf{can} iterate over them with for loops.
    \item Mutable, you can use add(element), remove(element) methods.
    \item You can use \textbf{union, intersection, difference, symmetric difference} operations on them.
    \item To create an empty one: \texttt{set()}.
\end{itemize}

\subsubsection{Dictionaries}
        \begin{itemize}
            \item Collection of \textbf{key$-$value} \underline{pairs}.
            \item Keys of a dictionary are \textbf{unique}.
            \item In general, they are not \textbf{ordered}. 
            \subitem However, in Python 3.7 pairs are guaranteed to be in insertion order.
            \subitem This means we will get pairs in insertion order if we loop over one.
            \subitem However, you \underline{\textbf{cannot}} use \textbf{indexing/slicing}.
            \subitem But you \textbf{can} iterate over them with \texttt{for} loops.
            \item To create an empty one: \texttt{\{\}} or \texttt{dict()}: empty dictionary
            \item To access values: \texttt{print(d[\textquotesingle one\textquotesingle ])} \# $\Rightarrow$ 1
        \end{itemize}
        
\subsection{Error/Exception Handling}
\subsubsection{Syntax Errors}
Errors that you get when you have a \textbf{syntactically incorrect} piece of code. i.e. \underline{Code that doesn't follow the rules of coding in Phython}.
\begin{lstlisting}
    for i in range(100)
    print(i)
    # SyntaxError: invalid syntax
    
    while True:
    print('Hello')
    # IndentationError: expected an indented block
\end{lstlisting}{}
    
\subsubsection{Runtime Errors}
When a statement is \textbf{syntactically correct}, this doesn't mean we are safe. Python interpreter will run the code in that case, but can still raise errors.
\begin{lstlisting}
    print(3 / 0)
    # ZeroDivisionError: division by zero
    
    int('hello')
    # ValueError: invalid literal for int() with base 10: 'hello'
    
    'hello'[2] = 'a'
    # TypeError: 'str' object does not support item assignment
\end{lstlisting}{}

\subsubsection{Try Except Blocks}
To be safe from these errors, we could put \texttt{if} checks everywhere. But it would be too much effort, and probably we cannot list every condition. The solution is \texttt{try-except-finally} blocks.

\begin{lstlisting}
    try:
        <risky-statement>
        <risky-statement>
        <risky-statement>
        ...
    except ValueError as valError:
        print('value error', valError)
    except (RuntimeError, TypeError, NameError):
        print('One of the above errors, but not ValueError')
    else:
        print('Here, do what you want to do if there are no errors.')
    finally:
        print('This part runs no matter what.')
\end{lstlisting}{}
It is true that Python throws these errors anyway. However, if you handle them like this:
\begin{itemize}
    \item You hold the power over deciding what to do if an error occurs.
    \item The program doesn't stop when it raises an exception. It just does what you wrote under the relevant except block and keeps running.
\end{itemize}{}

\textbf{Pro Tip:} You don't have to memorize the names of a whole list of errors! You can imagine what could go wrong, do the wrong thing and what the name of the error is from the Python itself!

\subsection{File Input/Output}
    Access to a \texttt{file object} using \texttt{\textbf{open(filename, mode=\textquotesingle r\textquotesingle )}} function
    \begin{itemize}
        \item \texttt{\textbf{filename}}: File name including the \textbf{file extension}. Ex: \textquotesingle data.txt\textquotesingle
        \subitem If you want to access/create a file outside of current \textbf{working directory}, you also need to include its path. Ex: \textquotesingle ./FolderName/data.txt\textquotesingle , \textquotesingle C:/Users/AUYSAL16/Desktop/data.txt\textquotesingle
        \item \textbf{\texttt{mode}} denotes how the file will be used. It is optional to declare mode, it has a default value of \texttt{w}:
        \begin{itemize}
            \item \textquotesingle r\textquotesingle : read mode, default
            \item \textquotesingle w\textquotesingle : write mode, overrides the file contents if it already exists
            \item \textquotesingle x\textquotesingle : create \& write mode, similar to write mode gives error if file already exists
            \item \textquotesingle a\textquotesingle : append mode, adds content to the end of file
        \end{itemize}
    \end{itemize}

\subsubsection{Reading file content}
    \begin{itemize}
        \item Open the file with read mode (which is already the default mode).
        \subitem Ex: \texttt{f = open(\textquotesingle my\_file.txt\textquotesingle)}
        \item \texttt{f.read()}: returns content of entire file as a string
        \item \texttt{f.readline(): returns a single line from file}
        \item \texttt{\textbf{for} line \textbf{in} f:} $\Rightarrow$ Iterate over all lines
        \item \texttt{list(f)}/\texttt{f.readlines()}: read file lines to a list
        \item \textbf{Always} close the file when you are done: \texttt{f.close()}
    \end{itemize}

\subsubsection{Creating/modifying files}
    \begin{itemize}
        \item Open the file with a write enabled mode, e.g, \texttt{w, x, a}
        \subitem Ex: \texttt{f = open(\textquotesingle my\_file\textquotesingle ,\textquotesingle w\textquotesingle )}
        \item Use \texttt{f.write(string)} to write to file.
        \subitem \underline{Warning!} \texttt{file.write()} method \textbf{only} takes \texttt{\textbf{str}} values!
        \item Close the file when you are done.
        \item \texttt{f.close()}
    \end{itemize}
        
\section{Exercises}
\subsection{Catch Me If You Can}
Our program is under attack. Guard our honor with necessary Try/Except blocks!
\textbf{Possibilities are: }
\begin{itemize}
    \item NameError
    \item SyntaxError
    \item ValueError
    \item ZeroDivisionError
    \item IndexError
    \item TypeError
    
\end{itemize}
Also, for the "faculties" part, add some code to the program so that it will ask for an index until the input is a number.

\subsection{Registrar's Office}
You are a work and study student for the Registrar's Office, help manage Kusis.
\begin{enumerate}
    \item Oops, our database is not updated since last Fall, So correct the instructor for Engr 200 by changing it to Lerzan Ormeci.
    \item Python Section 2 is missing 2 students, find and add them to the class set. 
    \item Chem 103 and Econ 201 students will attend an interdisciplinary seminar, find the number of students who will be attending this seminar.
    \item Math 205 class needs to be rescheduled, determine which time slot it can be placed without a time conflict.
\end{enumerate}
\newpage
\begin{lstlisting}[language={}]
Output:

Chem 103	Sarp Kaya	 {'Gamze', 'Ata', 'Ayse', 'Mahsa', 'Furkan'}
Engr 200	Lerzan Ormeci	 {'Ahmet', 'Canan', 'Furkan', 'Gonca'}
Econ 201	Seda Ertac	 {'Gokce', 'Emirhan', 'Ayse', 'Abdullah', 'Meva'}
Math 205	Nadim Rustom	 {'Ahmet', 'Zeynep', 'Mahsa', 'Ilayda'}
Python Section 2	Hasan Can Aslan	 {'Gokce', 'Gamze', 'Emirhan', 'Ahmet', 'Meva', 'Ayse', 'Abdullah', 'Ilayda', 'Mahsa', 'Canan', 'Furkan', 'Gonca'}

{'Ata', 'Zeynep'}

{'Chem 103': ('Sarp Kaya', {'Gamze', 'Ata', 'Ayse', 'Mahsa', 'Furkan'}), 'Engr 200': ('Lerzan Ormeci', {'Ahmet', 'Canan', 'Furkan', 'Gonca'}), 'Econ 201': ('Seda Ertac', {'Gokce', 'Emirhan', 'Ayse', 'Abdullah', 'Meva'}), 'Math 205': ('Nadim Rustom', {'Ahmet', 'Zeynep', 'Mahsa', 'Ilayda'}), 'Python Section 2': ('Hasan Can Aslan', {'Gokce', 'Emirhan', 'Ata', 'Ayse', 'Abdullah', 'Mahsa', 'Canan', 'Furkan', 'Gamze', 'Ahmet', 'Ilayda', 'Meva', 'Gonca', 'Zeynep'})}

9

['Econ 201']

\end{lstlisting}
\subsection{World Happiness Report}
You are given the dataset of World Happiness Report 2019, in a separate file named data.csv.
\begin{enumerate}
    \item Try to understand how data stored in data.csv file, and think about how to parse data. \newline
        * CSV is a flat file format describing values in a table. Each record consists of M values, separated by commas. The last value is followed by a new line instead of a comma.
    \item Try to understand what below code does:
    \begin{lstlisting}
with open('data.csv', 'r') as data_file:
    lines = data_file.readlines()
    # why did we use strip and split?
    columns = tuple(lines[0].strip().split(',')[1:])
    for row in lines[1:]:
        row_data = row.strip().split(',')
        data[row_data[0]] = tuple(row_data[1:])
    data_file.close()
    
\end{lstlisting}
\item Create a new file named corruption.csv that contains only country names and corruption statistics.
\end{enumerate}

\end{document}
